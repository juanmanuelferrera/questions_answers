% Created 2026-01-20 Tue 17:32
% Intended LaTeX compiler: pdflatex
\documentclass[11pt]{article}
\usepackage[utf8]{inputenc}
\usepackage[T1]{fontenc}
\usepackage{graphicx}
\usepackage{longtable}
\usepackage{wrapfig}
\usepackage{rotating}
\usepackage[normalem]{ulem}
\usepackage{amsmath}
\usepackage{amssymb}
\usepackage{capt-of}
\usepackage{hyperref}
\author{{[}Por definir]}
\date{2026}
\title{Grandes Preguntas - Capítulo de Muestra\\\medskip
\large Prólogo y Capítulo 1}
\hypersetup{
 pdfauthor={{[}Por definir]},
 pdftitle={Grandes Preguntas - Capítulo de Muestra},
 pdfkeywords={},
 pdfsubject={},
 pdfcreator={Emacs 30.2 (Org mode 9.7.11)}, 
 pdflang={English}}
\begin{document}

\maketitle
\tableofcontents

\section{PRÓLOGO: La Caja}
\label{sec:org6619755}

El notario tenía las manos manchadas de tinta azul. Elena lo notó mientras él buscaba entre los papeles, y ese detalle absurdo —las manos azules de un hombre en traje gris— fue lo que la mantuvo anclada a la realidad durante los siguientes veinte minutos.

—Su abuelo fue muy específico —dijo el hombre, ajustándose las gafas—. El apartamento pasa a su hermano Pablo. Las cuentas se dividen entre los dos. Pero esto\ldots{} —levantó una caja de cartón del tamaño de una caja de zapatos— esto es solo para usted.

Elena miró la caja. Estaba sellada con cinta adhesiva amarillenta, del tipo que se usaba hace décadas. En la tapa, con la letra temblorosa de su abuelo, decía:

\emph{Para Elena. Cuando esté lista.}

—¿Cuándo la dejó aquí?

—Hace quince años. En 2011. Vino personalmente. Me hizo prometer que se la entregaría solo a usted, solo después de su muerte, y solo si usted venía a buscarla.

—¿Y si no hubiera venido?

El notario se encogió de hombros.

—Tenía instrucciones de destruirla después de un año.

Elena tomó la caja. Pesaba menos de lo que esperaba. ¿Qué guardaba un hombre durante quince años para su nieta? ¿Qué no podía decirle en vida?

No la abrió en la notaría. Ni en el taxi. Ni siquiera cuando llegó a su apartamento y se sentó en el sofá con las luces apagadas y el ruido de Madrid filtrándose por la ventana.

La abrió a las tres de la mañana, cuando el insomnio la venció.

Dentro había:

Un cuaderno de cuero negro, gastado en las esquinas.

Una libreta pequeña con nombres y direcciones.

Un sobre manila con fotografías.

Y una carta.

\begin{center}
───────────────────
\end{center}

\emph{Querida Elena:}

\emph{Si estás leyendo esto, me fui antes de poder explicarte. Pasé cuarenta años buscando respuestas a preguntas que quizás no tienen respuesta. En este cuaderno están esas preguntas. En la libreta, las personas que me ayudaron a buscar.}

\emph{No encontré la verdad. Pero encontré algo mejor: encontré muchas verdades, cada una iluminando una parte del misterio.}

\emph{El primer nombre de la lista es Tomás. Está en Montserrat. Es el único que queda de los que me conocieron al principio, cuando yo era joven y estaba tan perdido como tú estás ahora.}

\emph{Sí, sé que estás perdida. Lo veo en tus ojos cada vez que me visitas. La misma mirada que yo tenía a tu edad.}

\emph{Ve a ver a Tomás. Él te contará lo que yo nunca me atreví a decirte en persona.}

\emph{Perdóname por el secreto.}

\emph{Con todo mi amor,}
\emph{Tu abuelo Antonio}

\begin{center}
───────────────────
\end{center}

Elena leyó la carta tres veces.

\emph{Perdóname por el secreto.}

¿Qué secreto guarda un hombre de ochenta y nueve años? ¿Qué no se había atrevido a decirle en treinta y ocho años de vida, en cientos de visitas, en miles de conversaciones?

Abrió el cuaderno. La primera página tenía una sola pregunta, escrita con letra firme —la letra del abuelo joven, antes del temblor:

\emph{¿Por qué existe algo en lugar de nada?}

Debajo, en tinta diferente, añadida años después:

\emph{Montserrat, 1962. Pregunté esto mientras pensaba en saltar.}

Elena cerró el cuaderno.

Su abuelo había pensado en suicidarse.

Su abuelo, el profesor tranquilo que le contaba historias de la India. El hombre que siempre tenía un libro en la mano y una sonrisa en los labios. El viudo sereno que nunca levantaba la voz.

Había querido saltar de una montaña.

Elena miró la libreta de direcciones. El primer nombre:

\emph{Hermano Tomás Mendoza}
\emph{Monasterio de Montserrat}
\emph{Barcelona}

Y una nota al margen: \emph{Si sigue vivo, tendrá 72 años. Fue mi alumno en 1975. Entró al monasterio después. Sabe todo.}

Elena no durmió esa noche.

A las siete de la mañana, reservó un billete de tren a Barcelona.
\section{CAPÍTULO 1: La Montaña}
\label{sec:org62b007b}

\subsection{I.}
\label{sec:org44432d5}

El tren salía a las 7:15 de Atocha. Elena llegó con veinte minutos de sobra y se sentó en un banco a mirar a la gente pasar.

Familias con maletas. Ejecutivos con portátiles. Estudiantes con mochilas. Todos sabían a dónde iban. Todos tenían un propósito.

Ella iba a un monasterio a preguntarle a un monje por qué su abuelo había querido matarse sesenta años atrás.

El absurdo de la situación la golpeó de pronto, y casi se rio. Casi. Hacía meses que no se reía de nada.

Desde el divorcio, para ser exacta. O quizás desde antes. Quizás desde que despidieron a la mitad de la redacción y ella empezó a sentir que su trabajo —escribir noticias que nadie leía sobre cosas que a nadie importaban— era una forma lenta de desaparecer.

O quizás desde que su madre murió y Elena descubrió que podías perder a una persona y seguir funcionando, seguir comiendo, seguir durmiendo, como si nada hubiera cambiado. Como si la pérdida fuera simplemente otro dato que el cuerpo procesaba y archivaba.

Eso era lo que más la asustaba: no el dolor, sino la ausencia de dolor. La capacidad de seguir adelante sin sentir nada.

El tren llegó. Elena subió.

Barcelona estaba a dos horas y media. Montserrat, a una hora más en el tren cremallera que subía la montaña.

Tenía tiempo para leer el cuaderno del abuelo.
\subsection{II.}
\label{sec:org8587a30}

Las primeras páginas eran de 1962. La letra era firme, casi agresiva. El abuelo tenía veintisiete años.

\begin{quote}
\emph{15 de marzo, 1962}

\emph{Dejé el seminario. Siete años de estudio para nada. Le dije al padre rector que ya no creía en Dios y él me dijo que la fe no era un sentimiento, era una decisión. Le respondí que no podía decidir creer en algo que me parecía absurdo.}

\emph{Me miró con lástima. Eso fue lo peor. No con enojo, no con decepción. Con lástima.}

\emph{"Ya volverás", dijo.}

\emph{No volveré.}
\end{quote}

Elena pasó las páginas. Había saltos de semanas, a veces meses.

\begin{quote}
\emph{2 de junio, 1962}

\emph{Si Dios no existe, ¿por qué existe algo? ¿Por qué hay un mundo en lugar de la nada?}

\emph{Antes la respuesta era simple: porque Dios lo creó.}

\emph{Ahora no tengo respuesta. Y sin respuesta, ¿qué sentido tiene?}
\end{quote}

\begin{quote}
\emph{18 de agosto, 1962}

\emph{Fui a Montserrat. No sé por qué. Quizás porque es el lugar más alto que conozco.}

\emph{Me senté en el mirador de Sant Joan. Mil metros de caída. Pensé: sería tan fácil.}

\emph{No salté. No por valentía. Por cobardía. O quizás porque una parte de mí todavía quiere saber.}

\emph{¿Por qué existe algo en lugar de nada?}

\emph{Si encuentro la respuesta, quizás pueda vivir.}
\end{quote}

Elena cerró el cuaderno. Las manos le temblaban.

Afuera, el paisaje había cambiado. Ya no eran los suburbios de Madrid sino los campos de Aragón, amarillos bajo el sol de octubre.

Su abuelo había ido a Montserrat a morir. Y no había muerto. Algo lo había detenido.

¿Qué?
\subsection{III.}
\label{sec:org2234419}

El monasterio de Montserrat estaba enclavado en la roca como si hubiera crecido de ella. Las agujas de piedra se alzaban contra el cielo azul, y abajo, muy abajo, el valle se extendía hasta donde alcanzaba la vista.

Elena entendió por qué el abuelo había venido aquí. Era el tipo de lugar que te hacía sentir pequeño. Insignificante. Un punto en la inmensidad.

El tipo de lugar donde las preguntas grandes parecían tener sentido.

Preguntó por el Hermano Tomás en la recepción de peregrinos. El joven monje que la atendió la miró con curiosidad.

—¿Tiene cita?

—No. Pero\ldots{} —Elena dudó—. Dígale que soy la nieta de Antonio Vidal.

El monje asintió y desapareció por un pasillo de piedra.

Elena esperó. En la pared había un cuadro de la Virgen de Montserrat, la Moreneta, con su cara negra y serena. De niña, el abuelo la había traído aquí una vez. Elena tenía siete u ocho años. Recordaba que él se había quedado mucho tiempo mirando esa imagen, en silencio, y ella se había aburrido y había tirado de su mano.

—Abuelo, vámonos.

—Un momento, cariño. Estoy hablando.

—¿Con quién?

—Con alguien que ya no está.

Elena no había entendido entonces. Ahora, treinta años después, mirando la misma imagen, se preguntó con quién hablaría el abuelo. ¿Con Dios? ¿Con alguien que había perdido?

—¿Elena?

Se volvió. Un hombre mayor estaba en el umbral del pasillo. Llevaba el hábito negro de los benedictinos y caminaba con la ayuda de un bastón. Tenía el pelo blanco, la cara surcada de arrugas, y unos ojos de un azul sorprendentemente claro.

—Soy Tomás —dijo—. Conocí a tu abuelo hace casi cincuenta años. —Hizo una pausa—. Te pareces a él. Los mismos ojos.

Elena sintió que algo se aflojaba en su pecho. No sabía qué había esperado —quizás hostilidad, quizás indiferencia—, pero no esto. No esta mirada de reconocimiento, como si el monje hubiera estado esperándola.

—Ven —dijo Tomás—. Tenemos mucho de qué hablar.
\subsection{IV.}
\label{sec:org136df78}

El despacho de Tomás era pequeño y austero: una cama, un escritorio, un crucifijo en la pared, y una ventana que daba al precipicio. En el escritorio había una fotografía enmarcada. Elena la reconoció al instante: su abuelo joven, con el pelo negro y una sonrisa que ella nunca le había visto. A su lado, un Tomás de quizás treinta años, también sonriendo.

—1975 —dijo Tomás, siguiendo su mirada—. El año que me hice monje. Tu abuelo vino a mi ordenación. Fue el único de mis profesores que vino.

—¿Era su profesor?

—De filosofía de la religión. Universidad Complutense. —Tomás se sentó con dificultad en una silla y le indicó a Elena que hiciera lo mismo—. Fue el mejor profesor que tuve. No porque supiera las respuestas, sino porque hacía las preguntas correctas.

Elena sacó el cuaderno del abuelo.

—Encontré esto. Él escribió que usted "sabe todo". ¿Todo sobre qué?

Tomás miró el cuaderno. Sus ojos se humedecieron.

—¿Te contó alguna vez por qué dejó el seminario?

—Dijo que había perdido la fe.

—Es verdad. Pero no es toda la verdad. —Tomás respiró hondo—. Tu abuelo no solo perdió la fe. Perdió las ganas de vivir. Vino aquí en el verano de 1962. Yo era novicio entonces, tenía dieciocho años. Lo encontré en el mirador de Sant Joan, sentado en el borde. Mirando hacia abajo.

Elena sintió el frío de la montaña entrar por la ventana.

—¿Qué hizo?

—Me senté a su lado. No dije nada. Solo me senté. —Tomás sonrió levemente—. A veces eso es lo único que puedes hacer. Estar presente.

—¿Y funcionó?

—No sé si "funcionó" es la palabra correcta. Después de un rato, él me preguntó: "¿Tú crees en Dios?" Le dije que sí. Me preguntó por qué. Le dije que no sabía, que simplemente creía. —Tomás hizo una pausa—. Me dijo que él había creído toda su vida, y que de pronto había dejado de creer, y que sin esa creencia no sabía cómo seguir viviendo.

—¿Qué le respondió?

—Le pregunté si no creer en Dios significaba que no podía creer en nada. Me miró como si le hubiera hablado en otro idioma. "¿En qué más podría creer?", me preguntó.

Elena esperó.

—Le dije: "En la pregunta. Mientras sigas preguntando, sigues vivo."
\subsection{V.}
\label{sec:orgdbd987e}

Caminaron por el claustro mientras el sol de la tarde alargaba las sombras de las columnas. Tomás caminaba despacio, apoyándose en el bastón, pero su voz era firme.

—Tu abuelo se quedó un mes aquí. No como peregrino, sino como huésped. Ayudaba en la biblioteca, hablaba con los monjes, asistía a los oficios aunque no comulgaba. Creo que estaba buscando algo.

—¿Encontró algo?

—No aquí. —Tomás se detuvo junto a una fuente seca—. Un día llegó una carta de Marruecos. De un hombre llamado Ibrahim al-Fassi. No sé cómo se habían conocido; tu abuelo nunca me lo explicó. Pero después de leer esa carta, todo cambió. Me dijo: "Tomás, creo que he estado buscando a Dios en el lugar equivocado."

—¿Qué significa eso?

—No lo sé. Nunca me lo explicó. Solo sé que se fue a Marruecos y volvió transformado. —Tomás la miró—. Pasaron muchos años antes de que nos viéramos otra vez. Cuando fui su alumno, en los setenta, ya era otro hombre. Seguía sin creer en el Dios católico, pero creía en\ldots{} algo. Algo que había encontrado en sus viajes.

Elena pensó en la libreta de direcciones. Doce nombres. Doce países.

—¿Sabe qué encontró?

—Solo sé lo que me dijo la última vez que lo vi, hace tres años. Vino a despedirse. Sabía que le quedaba poco tiempo. Me dijo: "Tomás, nunca encontré la respuesta. Pero encontré las preguntas correctas. Y encontré personas que me ayudaron a hacerlas."

—¿Le dijo algo sobre mí?

Tomás asintió.

—Me dijo: "Mi nieta Elena está perdida. Como yo lo estuve. Si algún día viene a buscarte, cuéntale todo. Y después mándala con Ibrahim. Él le devolverá a Dios. Pero no al Dios que ella espera."
\subsection{VI.}
\label{sec:org670e183}

Esa noche, Elena se quedó en la hospedería del monasterio. La habitación era pequeña —una cama, un lavabo, un crucifijo— y el silencio era tan profundo que podía oír su propia respiración.

No podía dormir.

Se levantó y se sentó junto a la ventana. Afuera, la montaña era una masa negra contra el cielo estrellado. En algún lugar allá abajo, sesenta años atrás, su abuelo había pensado en saltar.

Y no había saltado.

Porque un chico de dieciocho años se había sentado a su lado y le había dicho que creyera en la pregunta.

Elena abrió el cuaderno del abuelo en una página al azar.

\begin{quote}
\emph{12 de octubre, 1968. Fez.}

\emph{Ibrahim me preguntó por qué buscaba a Dios con la cabeza cuando debería buscarlo con el corazón.}

\emph{Le dije que el corazón engaña. Me dijo que la cabeza también. "Entonces, ¿cómo sé qué es verdad?", pregunté.}

\emph{Se rio. "La verdad no se sabe. Se vive."}

\emph{No entendí. Pero quiero entender.}
\end{quote}

Elena cerró el cuaderno.

Mañana volvería a Madrid. Pero no para quedarse.

Tenía que ir a Fez. Tenía que encontrar a Ibrahim al-Fassi —si seguía vivo— y preguntarle qué había hecho con su abuelo. Qué le había enseñado. Por qué el abuelo decía que Ibrahim le había "devuelto a Dios".

Y sobre todo, tenía que preguntarle qué significaba eso de buscar a Dios con el corazón.

Porque Elena había buscado muchas cosas en su vida —éxito, amor, sentido—, y siempre las había buscado con la cabeza. Con planes. Con lógica. Con esfuerzo.

Y había terminado a los treinta y ocho años, sola en una habitación de monasterio, sin trabajo, sin pareja, sin fe, leyendo el diario de un hombre muerto que había estado tan perdido como ella.

Quizás era hora de probar otra cosa.

Quizás era hora de buscar con el corazón.
\subsection{VII.}
\label{sec:org60b613f}

A la mañana siguiente, antes de irse, Elena volvió al despacho de Tomás. Él la esperaba con una taza de café y un sobre en las manos.

—Tu abuelo me dejó esto para ti. Me pidió que te lo diera solo si venías a buscarme. —Le entregó el sobre—. No sé qué contiene. Nunca lo abrí.

Elena miró el sobre. Tenía su nombre escrito con la letra del abuelo.

—¿Por qué no lo abrió?

—Porque no era para mí.

Elena guardó el sobre en el bolsillo de su chaqueta. Lo abriría después, cuando estuviera sola.

—Gracias —dijo—. Por todo.

Tomás la acompañó hasta la puerta del monasterio. El sol de la mañana bañaba la piedra de oro.

—Elena —dijo el monje—, antes de que te vayas, quiero decirte algo.

Ella esperó.

—Tu abuelo pasó cuarenta años buscando respuestas. No las encontró. Pero encontró algo más valioso: encontró las preguntas correctas. Y encontró personas que lo acompañaron mientras preguntaba. —Tomás puso una mano en su hombro—. No busques respuestas. Busca compañía para las preguntas.

Elena asintió. No confiaba en su voz.

—Y una cosa más —añadió Tomás—. Ibrahim al-Fassi debe de tener ochenta y cinco años. Si sigue vivo, está en Fez. En una zawiya sufí cerca de la medina. Tu abuelo decía que era el hombre más sabio que había conocido.

—¿Por qué?

—Porque era el único que admitía que no sabía nada.

\begin{center}
───────────────────
\end{center}

El tren bajaba la montaña. Elena miraba por la ventana cómo Montserrat se hacía pequeño, las agujas de piedra hundiéndose en el horizonte.

En su bolsillo llevaba el sobre del abuelo. En su mochila, el cuaderno y la libreta de direcciones.

No sabía qué encontraría en Fez. No sabía si Ibrahim seguía vivo. No sabía si encontraría respuestas o solo más preguntas.

Pero por primera vez en meses, quizás en años, sentía algo que había olvidado que existía.

Curiosidad.

Y eso, por ahora, era suficiente.

\begin{center}
\textbf{Fin del Capítulo 1}
\end{center}
\section{NOTA SOBRE EL CAPÍTULO}
\label{sec:orgc7a66f2}

\subsection{Elementos incluidos}
\label{sec:org2d98066}
\begin{enumerate}
\item \textbf{Gancho inicial}: El secreto del abuelo
\item \textbf{Backstory de Elena}: Crisis, divorcio, muerte de la madre, despido
\item \textbf{El misterio}: ¿Por qué el abuelo quiso suicidarse? ¿Qué lo salvó?
\item \textbf{Filosofía natural}: La pregunta "¿Por qué existe algo?" emerge de la crisis, no de una conferencia
\item \textbf{Personaje secundario vivo}: Tomás tiene historia propia, no solo existe para enseñar
\item \textbf{Gancho final}: Ibrahim al-Fassi, "le devolvió a Dios pero no al Dios que esperaba"
\item \textbf{Transformación sutil}: Elena pasa de "no siento nada" a "siento curiosidad"
\end{enumerate}
\subsection{Lo que NO hace este capítulo}
\label{sec:org8a97929}
\begin{itemize}
\item No da respuestas definitivas
\item No predica ni sermonea
\item No idealiza al abuelo (tenía secretos, pensó en suicidarse)
\item No hace a Elena especial (es una mujer común en crisis común)
\item No resuelve nada (solo abre preguntas)
\end{itemize}
\subsection{Conexión con el siguiente capítulo}
\label{sec:orgffc6894}
El Cap 2 empezará con Elena en el ferry a Marruecos, leyendo las notas del abuelo sobre Ibrahim, preguntándose qué significa "buscar a Dios con el corazón".
\end{document}

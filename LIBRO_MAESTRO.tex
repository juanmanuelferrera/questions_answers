% Created 2026-01-20 Tue 13:41
% Intended LaTeX compiler: pdflatex
\documentclass[11pt]{article}
\usepackage[utf8]{inputenc}
\usepackage[T1]{fontenc}
\usepackage{graphicx}
\usepackage{longtable}
\usepackage{wrapfig}
\usepackage{rotating}
\usepackage[normalem]{ulem}
\usepackage{amsmath}
\usepackage{amssymb}
\usepackage{capt-of}
\usepackage{hyperref}
\author{Jagannatha Mishra Dasa}
\date{2026}
\title{Grandes Preguntas, Diversas Respuestas\\\medskip
\large Documento Maestro para el Libro}
\hypersetup{
 pdfauthor={Jagannatha Mishra Dasa},
 pdftitle={Grandes Preguntas, Diversas Respuestas},
 pdfkeywords={},
 pdfsubject={},
 pdfcreator={Emacs 30.2 (Org mode 9.7.11)}, 
 pdflang={English}}
\begin{document}

\maketitle
\setcounter{tocdepth}{3}
\tableofcontents

\section{MARCO CONCEPTUAL: EL MODELO PROTHERO}
\label{sec:org7460c0e}

Cada tradición ve el mundo desde un PROBLEMA central y ofrece una SOLUCIÓN.
Esto no significa que ignoren otros problemas, sino que tienen un ÉNFASIS distintivo.

\begin{center}
\begin{tabular}{lllll}
Tradición & PROBLEMA & SOLUCIÓN & TÉCNICA & EJEMPLAR\\
\hline
Cristianismo & Pecado & Salvación & Fe, gracia, sacramentos & Jesús, santos\\
Islam & Orgullo & Sumisión (islam) & Los 5 pilares & Mahoma\\
Budismo & Sufrimiento & Despertar (nirvana) & Óctuple sendero & Buda\\
Judaísmo & Exilio & Retorno a Dios & Ley (halajá), estudio & Moisés, rabinos\\
Hinduismo & Olvido de Krishna & Amor devocional (prema) & Bhakti-yoga, canto & Chaitanya, Prabhupada\\
Confucianismo & Caos social & Armonía ritual & Ritos, educación, virtud & Confucio\\
Taoísmo & Artificio & Naturalidad (wu-wei) & Meditación, simplicidad & Lao Tzu\\
Estoicismo & Perturbación & Serenidad (ataraxia) & Control de juicios & Marco Aurelio\\
Existencialismo & Absurdo/inautent. & Autenticidad & Elección radical & Kierkegaard\\
Trad. Indígenas & Desequilibrio & Armonía con la tierra & Ceremonia, comunidad & Ancianos\\
\end{tabular}
\end{center}
\section{LOS 12 CAPÍTULOS DEL LIBRO}
\label{sec:orge1a2665}

\subsection{Capítulo 1: ¿Qué Existe Realmente?}
\label{sec:org46824e3}
\subsubsection{El problema humano}
\label{sec:org10e3fba}
Despertamos en un mundo que no pedimos. ¿Qué es todo esto? ¿Es real o ilusión?
¿Hay algo más allá de lo que vemos?
\subsubsection{Coincidencias entre tradiciones}
\label{sec:org523fc34}
\begin{itemize}
\item Todas reconocen una \textbf{realidad que trasciende} lo aparente
\item Todas afirman que hay \textbf{más de lo que percibimos} con los sentidos
\item Todas proponen alguna forma de \textbf{conocimiento más profundo}
\end{itemize}
\subsubsection{Las respuestas distintivas}
\label{sec:org0a214d4}

\begin{enumerate}
\item Cristianismo: Jerarquía del ser
\label{sec:org74c50cd}
\begin{quote}
"Dios es el Ser mismo (ipsum esse subsistens). Todo lo demás existe por participación
en Él. Hay una cadena: Dios → ángeles → humanos → animales → materia.
Todo tiene su lugar en el orden divino."
\end{quote}
\begin{itemize}
\item PROBLEMA: Sin Dios, la existencia carece de fundamento
\item SOLUCIÓN: Reconocer a Dios como fuente de todo ser
\end{itemize}
\item Hinduismo: Krishna es la fuente de todo
\label{sec:org6858223}
\begin{quote}
"Krishna es la Verdad Absoluta, la fuente de todo lo que existe.
Él tiene tres energías: la espiritual, la material, y las almas.
Todo es real —nada es ilusión total— pero hay jerarquía: lo espiritual
es eterno, lo material es temporal. El error no es que el mundo sea
falso, sino que lo tomemos como lo único real."
\end{quote}
\begin{itemize}
\item PROBLEMA: Pensamos que la materia es todo lo que existe
\item SOLUCIÓN: Reconocer a Krishna como fuente y refugio de todo
\end{itemize}
\item Budismo: Vacuidad interdependiente
\label{sec:org30e9c2a}
\begin{quote}
"Nada existe por sí mismo. Todo surge por causas y condiciones (pratityasamutpada).
No hay sustancia permanente, ni en las cosas ni en ti. Esta vacuidad (sunyata)
no es la nada, sino la apertura infinita de posibilidades."
\end{quote}
\begin{itemize}
\item PROBLEMA: Creemos en un "yo" sólido que sufre
\item SOLUCIÓN: Ver la vacuidad libera del sufrimiento
\end{itemize}
\item Islam: Unidad divina absoluta
\label{sec:org4135972}
\begin{quote}
"La ilaha illa Allah: No hay dios sino Dios. Todo lo que existe es signo (ayat)
de Allah. El cosmos entero da testimonio de Su unidad (tawhid).
Existir es ser creado y sostenido por el Único."
\end{quote}
\begin{itemize}
\item PROBLEMA: El politeísmo y el ego dividen la realidad
\item SOLUCIÓN: Reconocer la Unidad absoluta de Dios
\end{itemize}
\item Judaísmo: Creación con propósito
\label{sec:org96afe54}
\begin{quote}
"En el principio, Dios creó\ldots{} La existencia no es accidente sino acto de voluntad
divina. El mundo está inacabado; los humanos somos socios de Dios en tikkun olam,
la reparación del mundo."
\end{quote}
\begin{itemize}
\item PROBLEMA: El mundo está roto, en exilio de su origen
\item SOLUCIÓN: Trabajar con Dios para restaurarlo
\end{itemize}
\item Taoísmo: El misterio innombrable
\label{sec:org6cf2830}
\begin{quote}
"El Tao que puede nombrarse no es el Tao eterno. Antes del cielo y la tierra,
había algo nebuloso y completo. Es la madre de todas las cosas.
No conozco su nombre; lo llamo Tao."
\end{quote}
\begin{itemize}
\item PROBLEMA: Las palabras y conceptos distorsionan la realidad
\item SOLUCIÓN: Fluir con el misterio sin forzar definiciones
\end{itemize}
\item Filosofía Occidental: La pregunta permanece abierta
\label{sec:org749b22e}
\begin{quote}
"¿Por qué hay algo en lugar de nada?" (Leibniz). La filosofía no tiene UNA respuesta,
sino múltiples: formas eternas (Platón), sustancia material (materialismo),
construcción mental (idealismo), existencia sin esencia (existencialismo).
\end{quote}
\begin{itemize}
\item PROBLEMA: La razón sola no puede alcanzar certeza última
\item SOLUCIÓN: Seguir preguntando con honestidad intelectual
\end{itemize}
\item Tradiciones Indígenas: Red sagrada
\label{sec:orgdbb7571}
\begin{quote}
"Todo está vivo y conectado. La roca, el río, el águila, el ancestro:
todos son parientes. El Gran Espíritu fluye a través de todo.
No estamos EN la naturaleza; SOMOS naturaleza."
\end{quote}
\begin{itemize}
\item PROBLEMA: La separación de la naturaleza trae desequilibrio
\item SOLUCIÓN: Vivir en armonía con todos los seres
\end{itemize}
\end{enumerate}
\subsubsection{Diferencias que enriquecen}
\label{sec:org59c50f6}
\begin{center}
\begin{tabular}{ll}
Tradición & Pregunta que te hace\\
\hline
Cristiano & ¿Reconoces que no te creaste a ti mismo?\\
Hindú & ¿Puedes amar a Dios como persona, no solo como concepto?\\
Budista & ¿Puedes soltar la necesidad de que algo sea permanente?\\
Musulmán & ¿Ves los signos de lo divino en todo lo que existe?\\
Judío & ¿Qué puedes hacer para reparar lo que está roto?\\
Taoísta & ¿Puedes abrazar el misterio sin necesitar explicarlo?\\
Filósofo & ¿Estás dispuesto a no tener certeza?\\
Indígena & ¿Te sientes parte de la red de la vida?\\
\end{tabular}
\end{center}
\subsubsection{Síntesis: Lo que aprendemos de todas}
\label{sec:org0ad44c1}
Cada tradición ilumina una faceta del misterio:
\begin{itemize}
\item La existencia tiene \textbf{profundidad} (no es solo superficie material)
\item Hay \textbf{conexión} entre todas las cosas
\item El ser humano puede \textbf{acceder} a esa profundidad
\item La \textbf{humildad} es la actitud apropiada ante el misterio
\end{itemize}
\subsection{Capítulo 2: ¿Quién Soy Yo Realmente?}
\label{sec:orgaa6570a}
\subsubsection{El problema humano}
\label{sec:org0ab1e97}
Mi cuerpo cambia, mis pensamientos cambian, mis emociones cambian.
¿Qué permanece? ¿Hay un "yo" verdadero bajo todas las máscaras?
\subsubsection{Coincidencias entre tradiciones}
\label{sec:org0294ac8}
\begin{itemize}
\item Todas distinguen entre el yo \textbf{superficial} y algo más \textbf{profundo}
\item Todas reconocen que la identidad es un \textbf{misterio}, no algo obvio
\item Todas proponen alguna forma de \textbf{autoconocimiento} como camino
\end{itemize}
\subsubsection{Las respuestas distintivas}
\label{sec:org61daa64}

\begin{enumerate}
\item Cristianismo: Imagen de Dios
\label{sec:org97d4814}
\begin{quote}
"Dios creó al ser humano a su imagen y semejanza. Tienes un alma inmortal
con intelecto y voluntad libre. Tu identidad más profunda es ser hijo/a de Dios,
llamado a participar de la naturaleza divina."
\end{quote}
\item Hinduismo: Parte eterna de Krishna
\label{sec:org638c73d}
\begin{quote}
"Tú no eres este cuerpo. Eres alma espiritual eterna (jiva),
parte integral de Krishna. Tu identidad no es fusionarte con Dios
ni desaparecer en Él, sino amarlo eternamente como Su sirviente,
Su amigo, o Su amado. Tienes una relación única con Krishna
que ningún otro ser tiene. Eso es lo que debes descubrir."
\end{quote}
\item Budismo: No-yo (anatta)
\label{sec:org303b22c}
\begin{quote}
"Buscas un yo permanente pero no lo encuentras. Solo hay cinco agregados
(skandhas): forma, sensación, percepción, formaciones mentales, consciencia.
Todos cambian. El 'yo' es una construcción útil, no una entidad real."
\end{quote}
\item Taoísmo: Fluir sin fijarse
\label{sec:org8617ce6}
\begin{quote}
"El sabio no tiene yo fijo. Es como el agua: toma la forma del recipiente.
No te aferres a una identidad; permite que la vida te moldee."
\end{quote}
\end{enumerate}
\subsubsection{Diferencias que enriquecen}
\label{sec:org359e12d}
\begin{center}
\begin{tabular}{ll}
Tradición & Lo que aporta\\
\hline
Cristiano & Dignidad intrínseca como imagen divina\\
Hindú & Relación eterna y única con Dios\\
Budista & Libertad de no estar atado a una identidad fija\\
Taoísta & Flexibilidad y adaptación sin perder el centro\\
\end{tabular}
\end{center}
\subsection{Capítulo 3: ¿Tengo Control Sobre Mi Vida?}
\label{sec:org8251483}
\subsubsection{El problema humano}
\label{sec:org6fd31ba}
A veces siento que elijo; otras veces siento que las circunstancias me arrastran.
¿Soy libre o estoy determinado? ¿Mis decisiones importan?
\subsubsection{Coincidencias}
\label{sec:orgc2bc3c3}
\begin{itemize}
\item Todas afirman que \textbf{las acciones tienen consecuencias}
\item Todas reconocen \textbf{limitaciones} a la libertad humana
\item Todas proponen \textbf{responsabilidad} moral de algún tipo
\end{itemize}
\subsubsection{Las respuestas distintivas}
\label{sec:orgb4df308}

\begin{enumerate}
\item Cristianismo: Libertad y gracia
\label{sec:org6133e31}
"Dios te creó libre, pero el pecado dañó esa libertad.
Necesitas la gracia divina para ser verdaderamente libre."
\item Islam: Decreto divino y responsabilidad
\label{sec:org160b00b}
"Allah decretó todo (qadar), pero tú eres responsable de tus elecciones.
Esto es un misterio que no debemos pretender resolver completamente."
\item Budismo: Libertad dentro del karma
\label{sec:org3df035d}
"Tus acciones pasadas condicionan el presente, pero no lo determinan.
En cada momento tienes la posibilidad de actuar con sabiduría."
\item Estoicismo: Control de lo interno
\label{sec:org39789b6}
"No controlas lo que te pasa, solo cómo respondes.
Distingue lo que depende de ti (tus juicios, acciones) de lo que no."
\item Hinduismo: Libertad en la rendición
\label{sec:orgd4f0322}
"Tienes libre albedrío limitado: puedes elegir hacia dónde dirigir
tu amor —hacia Krishna o hacia maya. Los resultados están bajo
el control de Krishna. Tu libertad está en la intención; Su control
está en el resultado. Actúa para Él y acepta lo que Él dé."
\end{enumerate}
\subsubsection{Diferencias que enriquecen}
\label{sec:org3c94bdf}
\begin{center}
\begin{tabular}{ll}
Tradición & Donde pone el énfasis\\
\hline
Cristianismo & La gracia como liberadora\\
Islam & Confianza en el decreto divino\\
Budismo & Liberarse del determinismo mental\\
Estoicismo & Soberanía sobre la respuesta interior\\
Hindú & Rendirse a Krishna libera de la ansiedad\\
\end{tabular}
\end{center}
\subsection{Capítulo 4: ¿Por Qué Sufrimos?}
\label{sec:orge0d79da}
\subsubsection{El problema humano}
\label{sec:orgf88b72a}
El dolor es inevitable. Pero, ¿tiene sentido? ¿Por qué existe el mal
si hay un Dios bueno? ¿Es el sufrimiento absurdo o significativo?
\subsubsection{Coincidencias}
\label{sec:orgc587789}
\begin{itemize}
\item Ninguna tradición dice "el sufrimiento no existe" o "no importa"
\item Todas ofrecen algún \textbf{marco de sentido} para el sufrimiento
\item Todas proponen un \textbf{camino de transformación}
\end{itemize}
\subsubsection{Las respuestas distintivas}
\label{sec:org3b62261}

\begin{enumerate}
\item Cristianismo: Sufrimiento redentor
\label{sec:orgb2e4de2}
"Cristo sufrió por nosotros. Unidos a su cruz, nuestro sufrimiento
tiene valor redentor. El dolor puede acercarnos a Dios."
\item Budismo: Sufrimiento por apego
\label{sec:org6cd008f}
"Dukkha (sufrimiento/insatisfacción) surge del deseo y el apego.
Elimina la causa y cesarás el efecto. El camino óctuple es la medicina."
\item Hinduismo: El olvido de Krishna
\label{sec:org38f1c7c}
"El sufrimiento existe porque olvidamos a Krishna. El alma espiritual,
que pertenece al mundo espiritual, sufre cuando busca felicidad separada
de Él. Es como un pez fuera del agua: sufre no porque el aire sea malo,
sino porque está en el elemento equivocado. El dolor te llama a volver."
\item Estoicismo: Sufrimiento como juicio
\label{sec:orgd389bfd}
"Las cosas no nos perturban, sino nuestras opiniones sobre ellas.
Cambia tu juicio y transformas tu experiencia."
\item Judaísmo: Lamento y esperanza
\label{sec:orgbc35f8a}
"No pretendemos entender todo el sufrimiento. Job nos enseña a lamentar
honestamente y confiar en Dios sin tener todas las respuestas."
\end{enumerate}
\subsubsection{Diferencias que enriquecen}
\label{sec:org50b04af}
\begin{center}
\begin{tabular}{ll}
Tradición & Pregunta que te hace ante el sufrimiento\\
\hline
Cristiano & ¿Cómo puede esto acercarme a Dios?\\
Budista & ¿A qué me estoy aferrando?\\
Estoico & ¿Qué está en mi control aquí?\\
Hindú & ¿Me está llamando este dolor a volver a casa?\\
Judío & ¿Puedo lamentar y aún tener esperanza?\\
\end{tabular}
\end{center}
\subsection{Capítulo 5: ¿Cómo Debo Vivir?}
\label{sec:orgb21d6ab}
\subsubsection{El problema humano}
\label{sec:org05607f6}
Tengo una vida. ¿Cómo debo vivirla? ¿Qué es lo correcto? ¿De dónde viene
la autoridad moral?
\subsubsection{Coincidencias}
\label{sec:org6213670}
\begin{itemize}
\item Todas valoran alguna forma de \textbf{compasión/benevolencia}
\item Todas reconocen \textbf{obligaciones hacia otros}
\item Todas proponen \textbf{virtudes} a cultivar
\end{itemize}
\subsubsection{Las respuestas distintivas}
\label{sec:org1f20e02}

\begin{enumerate}
\item Cristianismo: Amor a Dios y al prójimo
\label{sec:org54f3d66}
"Ama a Dios con todo tu corazón y al prójimo como a ti mismo.
La ley se resume en el amor."
\item Confucianismo: Relaciones armoniosas
\label{sec:orge6ffb2b}
"Cultiva ren (benevolencia) y li (ritual apropiado).
Las cinco relaciones bien ordenadas crean armonía social."
\item Budismo: Compasión universal
\label{sec:org9fac4f5}
"Evita dañar, cultiva el bien, purifica la mente.
La compasión (karuna) se extiende a todos los seres sintientes."
\item Estoicismo: Vivir según la naturaleza
\label{sec:orga25cd0c}
"Vive de acuerdo con la razón universal (logos).
Las virtudes cardinales: sabiduría, justicia, coraje, templanza."
\item Taoísmo: Wu-wei (no-acción forzada)
\label{sec:org4f4e645}
"No forces, no manipules. Actúa naturalmente, sin artificio.
El agua vence a la roca siendo suave."
\item Hinduismo: Vivir para Krishna
\label{sec:orgfcb3634}
"Vive para servir a Krishna en todo lo que hagas. Come para Él,
trabaja para Él, ama para Él. El devoto no huye de la vida;
la santifica. Haz de tu vida una ofrenda. Canta Sus nombres,
estudia Sus palabras, sirve a Sus devotos."
\end{enumerate}
\subsection{Capítulo 6: ¿Qué Podemos Conocer?}
\label{sec:org4f10612}
\subsection{Capítulo 7: ¿Es la Verdad Absoluta o Relativa?}
\label{sec:org7f0dd21}
\subsection{Capítulo 8: ¿Fe o Razón?}
\label{sec:org197ad6a}
\subsection{Capítulo 9: ¿Qué es el Tiempo?}
\label{sec:org4c3e112}
\subsection{Capítulo 10: ¿Hay Vida Después de la Muerte?}
\label{sec:org12562a8}
\subsection{Capítulo 11: ¿Qué es Dios?}
\label{sec:orgd58bf2a}
\subsection{Capítulo 12: ¿Tiene la Vida un Propósito?}
\label{sec:orgd2c1b18}
\section{ESTRUCTURA NARRATIVA (Estilo Life of Pi)}
\label{sec:org1bd79d0}

\subsection{Opción A: El Buscador}
\label{sec:orga40b211}
Un personaje en crisis existencial viaja (real o metafóricamente)
encontrando maestros de cada tradición. Cada capítulo = un encuentro.
\subsection{Opción B: La Mesa Redonda}
\label{sec:org41b5801}
Un simposio imaginario donde representantes de cada tradición
discuten las grandes preguntas. Formato de diálogo.
\subsection{Opción C: El Heredero}
\label{sec:org7e098a4}
Un personaje hereda textos de un abuelo filósofo. Cada capítulo
descubre un nuevo texto de una tradición diferente.
\section{ELEMENTOS LITERARIOS A INCLUIR}
\label{sec:orgcd43d8e}

\subsection{Recursos por capítulo}
\label{sec:org58f34e7}
\begin{itemize}
\item \textbf{Apertura}: Escena concreta o pregunta provocadora
\item \textbf{Voces}: Citas directas de cada tradición (con personalidad)
\item \textbf{Tablas}: Comparación visual Problema/Solución
\item \textbf{Preguntas}: Lo que cada tradición te pregunta
\item \textbf{Síntesis}: Lo que aprendemos de todas juntas
\item \textbf{Cierre}: Reflexión o momento narrativo
\end{itemize}
\subsection{Voz del autor}
\label{sec:orgeb719c8}
\begin{itemize}
\item Personal pero no egocéntrica
\item Respetuosa de todas las tradiciones sin falsa neutralidad
\item Capaz de asombro genuino
\item Honesta sobre incertidumbres
\end{itemize}
\section{PRÓXIMOS PASOS}
\label{sec:org53a0578}

\begin{enumerate}
\item{$\square$} Completar los capítulos 6-12 con el mismo formato
\item{$\square$} Elegir estructura narrativa (A, B, o C)
\item{$\square$} Desarrollar personajes/voz narrativa
\item{$\square$} Escribir aperturas y cierres de cada capítulo
\item{$\square$} Revisar y pulir las citas de cada tradición
\item{$\square$} Agregar preguntas de reflexión al final
\end{enumerate}
\section{APÉNDICE: PARA QUIEN QUIERA PROFUNDIZAR}
\label{sec:org1c91aa5}

\subsection{Las escuelas del Hinduismo}
\label{sec:orgd740621}

El hinduismo no es una religión monolítica sino una familia de tradiciones.
Dos grandes corrientes filosóficas:

\begin{center}
\begin{tabular}{lll}
Aspecto & Advaita (Impersonal) & Vaishnavismo (Personal)\\
\hline
Dios & Brahman sin forma & Krishna, Persona Suprema\\
El alma & "Tú ERES Dios" (identidad total) & "Eres PARTE de Dios" (relación eterna)\\
El mundo & Ilusión (maya) & Real pero temporal\\
La meta & Fusionarse, perder individualidad & Amar eternamente, conservar individualidad\\
El camino & Conocimiento (jnana) & Devoción amorosa (bhakti)\\
Frase clave & "Tat tvam asi" (Tú eres Eso) & "Jivera svarupa haya krishnera nitya dasa" (El alma es sirviente eterno de Krishna)\\
\end{tabular}
\end{center}

En este libro, el personaje Gopal Das representa la corriente devocional (Vaishnavismo),
que enfatiza la relación personal con Dios. Otras escuelas hindúes ofrecerían respuestas
diferentes a las mismas preguntas.
\subsection{Las ramas del Budismo}
\label{sec:org429346a}

\begin{center}
\begin{tabular}{lll}
Rama & Énfasis & Región principal\\
\hline
Theravada & Liberación individual, monástico & Sudeste asiático\\
Mahayana & Compasión universal, bodhisattvas & China, Corea, Japón\\
Vajrayana & Prácticas tántricas, rituales & Tibet, Mongolia\\
\end{tabular}
\end{center}

En este libro, Tenzin Wangmo representa el budismo tibetano (Vajrayana),
y el Maestro Kenji representa el Zen (rama del Mahayana).
\subsection{Las corrientes del Islam}
\label{sec:org0d5cd36}

\begin{center}
\begin{tabular}{ll}
Corriente & Énfasis\\
\hline
Sunismo & Tradición del Profeta, consenso comunitario\\
Chiísmo & Autoridad de los Imanes, descendientes de Alí\\
Sufismo & Dimensión mística, amor a Dios\\
\end{tabular}
\end{center}

En este libro, Ibrahim al-Fassi representa el Sufismo,
y la Dra. Elif Yilmaz representa un Islam académico y progresista.
\subsection{Las denominaciones del Cristianismo}
\label{sec:orgf6a435a}

\begin{center}
\begin{tabular}{ll}
Denominación & Énfasis\\
\hline
Catolicismo & Tradición, sacramentos, autoridad papal\\
Ortodoxia & Liturgia, iconos, teología mística\\
Protestantismo & Escritura, fe personal, diversidad de formas\\
\end{tabular}
\end{center}

En este libro, el Hermano Tomás representa el monasticismo católico (benedictino).
\subsection{Las corrientes del Judaísmo}
\label{sec:org56b0abf}

\begin{center}
\begin{tabular}{ll}
Corriente & Énfasis\\
\hline
Ortodoxo & Observancia estricta de la halajá\\
Conservador & Equilibrio entre tradición y adaptación\\
Reformista & Adaptación a la modernidad\\
\end{tabular}
\end{center}

En este libro, el Rabino David Kohn representa el judaísmo conservador.
\end{document}
